\chapter{Metodologia}

Este capítulo descreve a metodologia utilizada no desenvolvimento desta monografia. O capítulo inicia-se com a descrição dos dados (Seção \ref{sec:dados}). Após, é relatado a parametrização dos algoritmos empregados (Seção \ref{sec:parametrizacao}). Por fim, as métricas de avaliação dos algoritmos são apresentadas (Seção \ref{sec:metricas_avaliacao}).

\section{Os dados}
\label{sec:dados}
Esta monografia foi concebida em parceria com o Hospital das Clínicas de Porto Alegre (HCPA), da Universidade Federal do Rio Grande do Sul (UFRGS). Os dados utilizados são de dependentes químicos internados de forma voluntária na Unidade de Internação para Tratamento de Alcoolismo e Dependência Química do HCPA. 

\section{Parametrização dos algoritmos}
\label{sec:parametrizacao}
Quando se trata de uma pesquisa científica, é importante garantir a repetibilidade e reprodutibilidade dos experimentos. A tabela \hl{Tabela} mostra as configurações dos algoritmos utilizados. Tais algoritmos já foram discutidos em detalhes no capítulo \ref{cap:fundamentacao}.

\section{Avaliação dos Resultados}
\label{sec:metricas_avaliacao}
Uma etapa importante em qualquer tarefa de mineração de dados é avaliar os modelos aprendidos. As métricas de avaliação servem para o pesquisador tenha subsídios de escolher qual o melhor modelo para o problema em questão. Para o atual trabalho, as métricas de avaliação utilizadas foram:
\begin{itemize}
  \item Acurácia (ACC)
  \item Precisão (PRE)
  \item Revocação (REV)
  \item F1-Score (F1)
  \item Área abaixo da Curva ROC (AUC)
\end{itemize}

Essas métricas foram escolhidas por serem as mais comuns quando se trata de problemas de classificação. No entanto, a melhor métrica dependerá do domínio do problema.