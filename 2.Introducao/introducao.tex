\chapter{Introdução}
\label{cap:introducao}
Este capítulo introdutório inicia com a motivação (Seção \ref{sec:motivacao}). Após, é discutido a relação entre a tentativa de suicídio e o consumo de drogas (Seção \ref{sec:suicidio_e_drogas}). Por fim, é apresentado os objetivos gerais e específicos da monografia (Seção \ref{sec:objetivos}).

\section{Motivação}
\label{sec:motivacao}
De acordo com um estudo realizado pela Organização Mundial da Saúde em 2014 \hl{(citar)}, mais de 1 milhão de pessoas tiram a própria vida por ano, essa é a segunda maior causa de mortes entre jovens de 15 a 29 anos de idade. No Brasil, em média, o número de suicídio é de 11 mil por ano, sendo a quarta maior causa de mortes entre os jovens do sexo masculino. Considerando apenas o grupo etário mais economicamente produtivo (15-44 anos), o suicídio é uma das três causas principais causa de morte em todo mundo, dessa forma, nota-se que também há impactos econômicos no ato de tirar a própria vida. 

Segundo um estudo realizado pelo Ministério da Saúde \hl{(citar)}, o número de suicídios no Brasil aumentou durante 16 anos consecutivos, entre os anos de 2000 a 2016. Os principais fatores de risco para o suicídio incluem doença mental ou física, abuso de álcool e drogas, mudança súbita na vida, como a perda de emprego, término de um casamento ou a combinação desses com outros acontecimentos. Para os amigos e familiares do suicida, o impacto é significativo, tanto imediatamente quanto a longo prazo. 

\section{O suicídio e o consumo de drogas}
\label{sec:suicidio_e_drogas}


Este projeto tem como base de estudo, dados de usuários de drogas que passaram pelo Hospital das Clínicas de Porto Alegre (HCPA/UFRGS) e por ventura, já haviam tentado cometer suicídio anteriormente.

\section{Objetivos}
\label{sec:objetivos}
O principal objetivo do atual projeto é utilizar técnicas de mineração de dados para criar um modelo preditivo a fim de predizer se um determinado paciente tentará cometer suicídio ou não. 

Como objetivo específico pretende-se descobrir os atributos que mais influenciam na tentativa de um suicídio.


