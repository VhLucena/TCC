\chapter{Introdução}
\label{cap:introducao}

De acordo com um estudo realizado pela Organização Mundial da Saúde em 2014, mais de 1 milhão de pessoas tiram a própria vida por ano, é a segunda maior causa de mortes entre jovens de 15 a 29 anos de idade. No Brasil, em média, o número de suicídio é de 11 mil por ano, sendo a quarta maior causa de mortes entre os jovens do sexo masculino. Considerando apenas o grupo etário mais economicamente produtivo (15-44 anos), o suicídio é uma das três causas principais causa de morte em todo mundo, dessa forma, nota-se que também há impactos econômicos no ato de tirar a própria vida. 

Segundo um estudo realizado pelo Ministério da Saúde, o número de suicídios no Brasil aumentou durante 16 anos consecutivos, entre os anos de 2000 a 2016. Os principais fatores de risco para o suicídio incluem doença mental ou física, abuso de álcool e drogas, mudança súbita na vida, como a perda de emprego, término de um casamento ou a combinação desses com outros acontecimentos. Para os amigos e familiares do suicida, o impacto é significativo, tanto imediatamente quanto a longo prazo. 

O Hospital das Clínicas de Porto Alegre (HCPA) é uma instituição pública e universitária, ligada ao Ministério da Educação e à Universidade Federal do Rio Grande do Sul (UFRGS).
A pesquisa produzida no HCPA, por sua vez, introduz novos conhecimentos, técnicas e tecnologias que beneficiam toda a sociedade, além de formar novas gerações de pesquisadores.

% ----------------------------------------------------------------------------------------
parágrafo 1: explicar o que são e como se relacionam: HCPA, UFRGS, Departamento de Psiquiatria e Medicina Legal, Programa de Pós-Graduação em Psiquiatria e Ciências do Comportamento, e o CPAD. Tem informações na Web e um resumo no texto do projeto enviado por email

parágrafo 2: resumir o projeto em que eu estou trabalhando com base no texto do projeto e no meu Plano de Trabalho enviados por email.

parágrafo 3: explicar que o foco do trabalho apresentado nesta monografia é contribuir com a construção dos modelos preditivos de tentativa de suicídio.
% ----------------------------------------------------------------------------------------


\section{Objetivos}
\label{sec:objetivos}
O principal objetivo do atual projeto é utilizar técnicas de mineração de dados para criar um modelo preditivo a fim de predizer se um determinado paciente tentará cometer suicídio ou não. 

Como objetivo específico pretende-se descobrir os atributos que mais influenciam na tentativa de um suicídio.

\section{Estrutura da monografia}
